\documentclass[final]{beamer}
%Poster Packages: 
%dimensions are in cm, scale is for font size
\usepackage[size=custom,height=91.4,width=121.9,scale=1.8]{beamerposter}
%General Packages:
\usepackage{amsmath,amsthm, amssymb, latexsym}
%TikZ:
\usepackage{tikz}
\usetikzlibrary{shapes,arrows}
\input{arrowsnew}

%Theme:
\mode<presentation>{\usetheme{I6pd2}}

\graphicspath{{figures/}}

%\newlength{\columnheight}
%\setlength{\columnheight}{88cm}

\title{\huge Modeling Intervention Strategies for United States TB Control}
\author{Jessica Ginepro, Emma Hartman, Ryo Kimura, Matthew McDermott, Colin
        Pawlowski, \& Dylan Shepardson}
\institute[MHC]{Mathematical Modeling Group, Mount Holyoke College, South
                Hadley, MA, USA}
\date[July 31, 2013]{July 31, 2013}
\begin{document}
%For the compartmental models:
\tikzstyle{compartment} = [rectangle, draw, fill=blue!20, text badly centered,
  node distance=3.8em, inner sep=0.4em, rounded corners]
\tikzstyle{compartment rate} = [right of=start, color=black, node distance=2.2em,
text width=3em]
%TODO: Increase arrow head size
\tikzstyle{line} = [draw, -latexnew, arrowhead = 7mm]


\begin{frame}
  \begin{columns}[T]
    \begin{column}{.3\textwidth}
      \begin{block}{Introduction}
        Epidemiological models offer insight into the structure of disease outbreaks and the merits of various interventions.  Compartmental differential equation models are a common model in which populations move between various health states, or compartments, according to predetermined rates.  This work is an extension of the Hill Model, a complex compartmental model of tuberculosis (TB) in the United States.
      \end{block}
      \vspace{1em}
      \begin{block}{The Basic Hill Model}
        \begin{columns}[T]
          \begin{column}{.5\textwidth}
            \begin{figure}[h]
              \begin{center}
                \includegraphics[width=\textwidth]{HillModelFlowChart}
              \end{center}
              \caption{The Hill Model schematic}
              \label{fig:hillFlow}
            \end{figure}
          \end{column}
          \begin{column}{.45\textwidth}
              \vspace{1.5em}
            Populations:
            \begin{itemize}
              \item US Born (USB) 
              \item Foreign Born (FB)
              %\item USB TB incidence declining
              %\item FB latent TB infection (LTBI) high
              %\item TB elimination in total population not projected by 2100
            \end{itemize}
            Individuals also leave the model due to natural death.
          \end{column}
        \end{columns}

        \vspace{1em}
        \begin{itemize}
          \item USB TB incidence rates are declining
          \item FB latent TB infection (LTBI) arrivals remain high
          \item TB elimination in total population not projected by 2100
        \end{itemize}
        \vspace{1em}
        \begin{columns}
          \begin{column}{.47\textwidth}
            \begin{figure}[h]
              \begin{center}
                \includegraphics[width=\textwidth] {incPlotSourced2}
              \end{center}
              \caption{The source population of US TB incidence}
              \label{fig:incPlotSourced}
            \end{figure}
          \end{column}
          \begin{column}{.47\textwidth}
            \begin{figure}[h]
              \begin{center}
                \includegraphics[width=\textwidth]{costPlotSourced2}
              \end{center}
              \caption{Source population of US HCS TB cost}
              \label{fig:incPlotTotal}
             \end{figure}
           \end{column}
         \end{columns}
       \end{block}
    \end{column}
    
    \begin{column}{.3\textwidth}
      %\vspace{-1.5em}
      \begin{block}{Analyzing US TB Reduction Strategies}
        \begin{itemize}
          \item Implemented in \texttt{R}, with various numerical DE solvers
          \item Tracks US Health Care System (HCS) cost
          \item Tracks statistics about various health states
        \end{itemize}
      \end{block}

      \begin{block}{Intervention Analysis}
        %\vspace{-1.6em}
        \begin{figure}[h]
          \begin{minipage}[c]{0.6\textwidth}
            \includegraphics[height=0.6\textwidth,width=\textwidth]{redEnLTBIIncGrouped}
          \end{minipage}
          \hspace{0.5em}
          \begin{minipage}[c]{0.35\textwidth}
            \caption{Incidence/million in USB, FB, and total populations,
                     given 0\%, 50\%, 75\%, or 100\% treatment of incoming
                     LTBI}
          \end{minipage}
          \label{fig:redEnLTBI_incidence}
        \end{figure}
        \begin{figure}[h]
          \begin{minipage}[c]{0.6\textwidth}
            \includegraphics[height=0.6\textwidth,width=\textwidth]{incLTBItrmtIncGrouped}
          \end{minipage}
          \hspace{0.5em}
          \begin{minipage}[c]{0.35\textwidth}
            \caption{Incidence/million in USB, FB, and total populations,
                     given 0\%, 100\%, or 300\% LTBI treatment increase}
          \end{minipage}
          \label{fig:incLTBItrmt_incidence}
        \end{figure}
      \end{block}
      \begin{block}{Economic Modeling}
        \begin{itemize}
          \item Tracks treatment costs for various disease states
          \item Estimates implementation cost of intervention
        \end{itemize}
        \begin{columns}[T]
          \begin{column}{.66\textwidth}
            \begin{figure}[h]
              \begin{center}
                \includegraphics[width=\textwidth,height=13cm]{EnLTBIRedGroupCost.pdf}
              \end{center}
              \caption{Cumulative implementation costs, US HCS savings, and net
                       US costs of LTBI arrival cure rates. Cost/case cured was
                       \$600, \$800, and \$1000 for 50\%, 75\%, and 100\%
                       cured}
              \label{fig:redEnLTBI_costs}
            \end{figure}
          \end{column}
          \begin{column}{.26\textwidth}
            Base HCS Costs:
            \begin{description}
              \item[Active TB:]\hfill \\ 
                \$14,014.90
              \item[LTBI:]\hfill \\ 
                \$403.45
            \end{description}
          \end{column}
        \end{columns}
      \end{block}
    \end{column}

    \begin{column}{.3\textwidth}
      %\vspace{-.5em}
      \begin{block}{An Agent-Based Implementation}
        Agent-based models capture disease dynamics on the individual level and
        reflect stochasticity and granularity lost in compartmental models.
        Agent-based counterparts to the Hill model were implemented in Netlogo
        and \texttt{C++}.
        \begin{columns}
          %\vspace{-2em}
          \begin{column}{.45\textwidth}
            \begin{figure}[h]
              \begin{center}
                \includegraphics[width=\textwidth]{NLHMinc}
              \end{center}
              \caption{Incidence/million for R and NetLogo models (12 runs, $\Delta t$ = 0.1, popConst = 100)}
              \label{fig:NLHMinc}
            \end{figure}
          \end{column}
          \begin{column}{.45\textwidth}
            \begin{figure}[h]
              \begin{center}
                \includegraphics[width=\textwidth]{finalRunSmall2}
              \end{center}
              \caption{Incidence/million for R and C++ models (2100 runs, $\Delta t$ = 0.01, popConst = 1)}
              \label{fig:finalRun}
            \end{figure}
          \end{column}
        \end{columns}
      \end{block}
      
      \begin{block}{Stochastic Models as a Measure of Variability}
        \begin{columns}
          %\vspace{-2em}
          \begin{column}{.45\textwidth}
            \begin{figure}[h]
              \begin{center}
                \includegraphics[width=\textwidth]{IN0dist}
              \end{center}
              \caption{Distribution of USB Incidence (C++) with fitted Normal curve}
              \label{fig:IN0dist}
            \end{figure}
          \end{column}
          \begin{column}{.45\textwidth}
            \begin{figure}[h]
              \begin{center}
                \includegraphics[width=\textwidth]{IN1dist}
              \end{center}
              \caption{Distribution of FB Incidence (C++) with fitted Normal curve}
              \label{fig:IN1dist}
            \end{figure}
          \end{column}
        \end{columns}
      \end{block}
      
      \vspace{1em}
      \begin{block}{References}
        Hill, A. N., Becerra, J. E., \& Castro, K. G. (2012). Modelling tuberculosis trends in the USA. Epidemiology and infection, 140(10), 1862.
      \end{block}
      
    \end{column}
  \end{columns}
\end{frame}
\end{document}
