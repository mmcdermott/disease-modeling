\documentclass[final]{beamer}
%Poster Packages: 
%dimensions are in cm, scale is for font size
\usepackage[size=custom,height=91.4,width=121.9,scale=1.8]{beamerposter}
%General Packages:
\usepackage{amsmath,amsthm, amssymb, latexsym}
%TikZ:
\usepackage{tikz}
\usetikzlibrary{shapes,arrows}
\input{arrowsnew}

%Theme:
\mode<presentation>{\usetheme{I6pd2}}

\graphicspath{{figures/}}

%\newlength{\columnheight}
%\setlength{\columnheight}{88cm}

\title{\huge Modeling Intervention Strategies for United States TB Control}
\author{Jessica Ginepro, Emma Hartman, Ryo Kimura, Matthew McDermott, Colin
        Pawlowski, \& Dylan Shepardson}
\institute[MHC]{Mathematical Modeling Group, Mount Holyoke College, South
                Hadley, MA, USA}
\date[July 31, 2013]{July 31, 2013}
\begin{document}
%For the compartmental models:
\tikzstyle{compartment} = [rectangle, draw, fill=blue!20, text badly centered,
  node distance=3.8em, inner sep=0.4em, rounded corners]
\tikzstyle{compartment rate} = [right of=start, color=black, node distance=2.2em,
text width=3em]
%TODO: Increase arrow head size
\tikzstyle{line} = [draw, -latexnew, arrowhead = 7mm]


\begin{frame}
  \begin{columns}
    \begin{column}{.3\textwidth}
      \vspace{-3.6em}
      \begin{block}{Introduction}
        Epidemiological models offer insight into the structure of disease
        outbreaks and the merits of various interventions. The most common
        epidemiological models are compartmental differential equation models,
        such as the SIR system, illustrated in figures~\ref{fig:SIRFlowchart}
        and~\ref{eq:SIRdes}. 
       %TODO equalize figure heights
        \begin{block}{}
          \begin{column}{.39\textwidth}
            \begin{figure}[h]
              \begin{center}
                \begin{tikzpicture}
                  \node [compartment] (S) {Susceptible (S)};
                  \node [compartment, below of=S] (I) {Infected (I)};
                  \node [compartment, below of=I] (R) {Recovered (R)};
                  \path [line] (S) -- node [compartment rate] {$\beta S(t)I(t)$} (I);
                  \path [line] (I) -- node [compartment rate] {$\gamma I(t)$} (R);
                \end{tikzpicture}
              \end{center}
              \caption{This flowchart depicts the standard SIR epidemiological
                       model. It is accomponanied by the system of differential
                       equations~\ref{eq:SIRdes}.}
              \label{fig:SIRFlowchart}
            \end{figure}
          \end{column}
          \begin{column}{.39\textwidth}
            \begin{figure}[h]
              \begin{align*}
                \frac{dS}{dt} &= -\beta S(t)I(t) \\
                \frac{dI}{dt} &= \beta S(t)I(t) - \gamma I(t) \\
                \frac{dR}{dt} &= \gamma I(t)\\
                N             &= S(t) + I(t) + R(t)\\
              \end{align*}
              \caption{The system of differential equations governing the SIR
                       model.}
              \label{eq:SIRdes}
            \end{figure}
          \end{column}
        \end{block}
      \end{block}
      
      \begin{block}{The Basic Hill Model}
        In order to model tuberculosis (TB) in the United States (US), Hill,
        Becerra, and Castro designed a complex compartmental model called the
        hill model. 
        \vspace{-2em}
        \begin{block}{}
          \begin{column}{.6\textwidth}
            \begin{figure}[h]
              \begin{center}
                \includegraphics[scale=.5]{HillModelFlowChart}
              \end{center}
              \caption{A flow chart representing the compartments of the Hill
                       Model.}
              \label{fig:hillFlow}
            \end{figure}
          \end{column}
          \begin{column}{.33\textwidth}
            Populations:
            \begin{itemize}
              \item US Born Individuals (USB) 
              \item Foreign Born Individuals (FB)
            \end{itemize}
            Individuals also leave the model due to natural death.
          \end{column}
        \end{block}
      \end{block}
    \end{column}
    
    \begin{column}{.3\textwidth}
      \vspace{-1.5em}
      \begin{block}{Analyzing US TB Reduction Strategies}
        \begin{itemize}
          \item Implemented in \texttt{R}, with various numerical DE solvers.
          \item Tracks US Health Care System (HCS) cost.
          \item Tracks statistics about various health states.
        \end{itemize}
        \begin{block}{Basic Hill Behaviour}
          \vspace{-1.6em}
          \begin{column}{.45\textwidth}
            \begin{figure}[h]
              \begin{center}
                \includegraphics[height=10cm,width=\textwidth]{incPlotSourced}
              \end{center}
              \caption{The source population of US TB incindecence}
              \label{fig:incPlotSourced}
            \end{figure}
          \end{column}
          \begin{column}{.45\textwidth}
            \begin{figure}[h]
              \begin{center}
                \includegraphics[height=10cm,width=\textwidth]{incPlotTotal}
              \end{center}
              \caption{New cases per year of various types of TB in the US.}
              \label{fig:incPlotTotal}
            \end{figure}
          \end{column}
        \end{block}
        \begin{block}{Intervention Analysis}
          \vspace{-1.6em}
          \begin{column}{.45\textwidth}
            \begin{figure}[h]
              \begin{center}
                \includegraphics[height=10cm,width=\textwidth]{incidencePlotRedEnLTBI}
              \end{center}
              \caption{Incidence/million in USB, FB, and total populations,
                       given 0\%, 50\%, 75\%, or 100\% treatment of incoming
                       LTBI.}
              \label{fig:redEnLTBI_incidence}
            \end{figure}
          \end{column}
          \begin{column}{.45\textwidth}
            \begin{figure}[h]
              \begin{center}
                \includegraphics[height=10cm,width=\textwidth]{incLTBItrmtIncGrouped}
              \end{center}
              \caption{Incidence/million in USB, FB, and total populations,
                       given 0\%, 100\%, or 300\% LTBI treatment increase.}
              \label{fig:incLTBItrmt_incidence}
            \end{figure}
          \end{column}
        \end{block}
      \end{block}
      \begin{block}{Economic Modeling}
        \begin{itemize}
          \item Tracks treatment costs for various disease states
          \item Estimates implementation cost of intervention
        \end{itemize}
        \begin{block}{}
          \vspace{-4em}
          \begin{column}{.66\textwidth}
            \begin{figure}[h]
              \begin{center}
                \includegraphics[width=\textwidth,height=13cm]{EnLTBIRedGroupCost.pdf}
              \end{center}
              \caption{Cumulative implementation costs, US HCS savings, and net
                       US costs of LTBI arrival cure rates. Cost/case cured was
                       \$600, \$800, and \$1000 for 50\%, 75\%, and 100\%
                       cured.}
              \label{fig:redEnLTBI_costs}
            \end{figure}
          \end{column}
          \begin{column}{.26\textwidth}
            Base HCS Costs:
            \begin{description}
              \item[Active TB:]\hfill \\ 
                \$14,014.90
              \item[LTBI:]\hfill \\ 
                \$403.45
            \end{description}
          \end{column}
        \end{block}
      \end{block}
    \end{column}
    \begin{column}{.3\textwidth}
      \vspace{-.5em}
      \begin{block}{An Agent Based Implementation}
        Agent based models capture disease dynamics on the individual level, and
        reflect stochasticity and granularity lost in compartmental models.
        Agent based counterparts to the Hill model were implemented in Netlogo
        and \texttt{c++}.
        \begin{block}{}
          \vspace{-2em}
          \begin{column}{.45\textwidth}
            \begin{figure}[h]
              \begin{center}
                \includegraphics[width=\textwidth]{NLHMinc}
              \end{center}
              \caption{Incidence/million for R and NetLogo models (12 runs, $\Delta t$ = 0.1, popConst = 100)}
              \label{fig:NLHMinc}
            \end{figure}
          \end{column}
          \begin{column}{.45\textwidth}
            \begin{figure}[h]
              \begin{center}
                \includegraphics[width=\textwidth]{finalRunSmall}
              \end{center}
              \caption{Incidence/million for R and C++ models (2100 runs, $\Delta t$ = 0.01, popConst = 1)}
              \label{fig:finalRun}
            \end{figure}
          \end{column}
        \end{block}
      \end{block}
      
      \begin{block}{Stochastic Models as a Measure of Variability}
        The stochastic model provides data on the variability of the results
        of the deterministic model.
        \begin{block}{}
          \begin{column}{.45\textwidth}
            \begin{figure}[h]
              \begin{center}
                \includegraphics[width=\textwidth]{IN0dist}
              \end{center}
              \caption{Distribution of USB Incidence (C++) with fitted Normal curve ($\mu = 0.233196, \sigma = 0.0165344$)}
              \label{fig:IN0dist}
            \end{figure}
          \end{column}
          \begin{column}{.45\textwidth}
            \begin{figure}[h]
              \begin{center}
                \includegraphics[width=\textwidth]{IN1dist}
              \end{center}
              \caption{Distribution of FB Incidence (C++) with fitted Normal curve ($\mu = 98.6087, \sigma = 0.525294$)}
              \label{fig:IN1dist}
            \end{figure}
          \end{column}
        \end{block}
      \end{block}
      
      \begin{block}{Future Extensions}
        \begin{itemize}
          \item Including contact structure
          \item Multi-drug resistant TB
          \item HIV
        \end{itemize}
      \end{block}
      
    \end{column}
  \end{columns}
\end{frame}
\end{document}
