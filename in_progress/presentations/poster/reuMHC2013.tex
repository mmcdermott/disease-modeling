\documentclass[final]{beamer}
%Poster Packages: 
%\usepackage[orientation=lanscape,size=a0]{beamerposter}
%dimensions are in cm, scale is for font size
\usepackage[size=custom,height=91.4,width=121.9,scale=1.8]{beamerposter}
%General Packages:
\usepackage{amsmath,amsthm, amssymb, latexsym}
%TikZ:
\usepackage{tikz}
\usetikzlibrary{shapes,arrows}
\input{arrowsnew}

%Theme:
\mode<presentation>{\usetheme{I6pd2}}

\graphicspath{{figures/}}

\newlength{\columnheight}
\setlength{\columnheight}{105cm}

\title{\huge Modeling Intervention Strategies for TB Control in the United States}
\author{Jessica Ginepro, Emma Hartman, Ryo Kimura, Matthew McDermott, Colin
        Pawlowski, \& Dylan Shepardson}
\institute[MHC]{Mathematical Modeling Group, Mount Holyoke College, South
                Hadley, MA, USA}
\date[July 31, 2013]{July 31, 2013}
\begin{document}
%For the compartmental models:
\tikzstyle{compartment} = [rectangle, draw, fill=blue!20, text badly centered,
  node distance=3.8em, inner sep=0.4em, rounded corners]
\tikzstyle{compartment rate} = [right of=start, color=black, node distance=2.2em,
text width=3em]
%TODO: Increase arrow head size
\tikzstyle{line} = [draw, -latexnew, arrowhead = 7mm]


\begin{frame}
  \begin{columns}
    \begin{column}{.3\textwidth}
      \vspace{-8em}
      \begin{block}{Introduction}
        Epidemiological models offer insight into the structure of disease
        outbreaks and the merits of various interventions. The most common
        epidemiological models are compartmental differential equation models,
        such as the SIR system, illustrated in figures~\ref{fig:SIRFlowchart}
        and~\ref{eq:SIRdes}. 
       %TODO equalize figure heights
        \vspace{-2em}
        \begin{block}{}
          \begin{column}{.39\textwidth}
            \begin{figure}[h]
              \begin{center}
                \begin{tikzpicture}
                  \node [compartment] (S) {Susceptible (S)};
                  \node [compartment, below of=S] (I) {Infected (I)};
                  \node [compartment, below of=I] (R) {Recovered (R)};
                  \path [line] (S) -- node [compartment rate] {$\beta S(t)I(t)$} (I);
                  \path [line] (I) -- node [compartment rate] {$\gamma I(t)$} (R);
                \end{tikzpicture}
              \end{center}
              \caption{This flowchart depicts the standard SIR epidemiological
                       model. It is accomponanied by the system of differential
                       equations~\ref{eq:SIRdes}.}
              \label{fig:SIRFlowchart}
            \end{figure}
          \end{column}
          \begin{column}{.39\textwidth}
            \begin{figure}[h]
              \begin{align*}
                \frac{dS}{dt} &= -\beta S(t)I(t) \\
                \frac{dI}{dt} &= \beta S(t)I(t) - \gamma I(t) \\
                \frac{dR}{dt} &= \gamma I(t)\\
                N             &= S(t) + I(t) + R(t)\\
              \end{align*}
              \caption{The system of differential equations governing the SIR
                       model.}
              \label{eq:SIRdes}
            \end{figure}
          \end{column}
        \end{block}
      \end{block}
      
      \begin{block}{The Basic Hill Model}
        In order to model tuberculosis (TB) in the United States (US), Hill,
        Becerra, and Castro designed a complex compartmental model called the
        hill model. 
        \vspace{-2em}
        \begin{block}{}
          \begin{column}{.6\textwidth}
            \begin{figure}[h]
              \begin{center}
                \includegraphics[scale=.5]{HillModelFlowChart}
              \end{center}
              \caption{A flow chart representing the compartments of the Hill
                       Model.}
              \label{fig:hillFlow}
            \end{figure}
          \end{column}
          \begin{column}{.33\textwidth}
            Populations:
            \begin{itemize}
              \item US Born Individuals (USB) 
              \item Foreign Born Individuals (FB)
            \end{itemize}
            Individuals also leave the model due to natural death.
          \end{column}
        \end{block}
        %For this implementation, various numerical solvers were implemented,
        %including an Eulerian Method, a Quadratic Method, a Fourth-order runge
        %kutta method, and the $\texttt{lsoda}$ routine. Once the basic Hill
        %model was implemented, further tracking capabilities were added as well
        %as economic modeling. This implementation tracks total active TB cases,
        %entering cases of LTBI, and treatment costs for active and latent disease
        %for all populations. Tracking was used to evaluate various interventions
        %explored in Hill, Becerra, and Castro's work, as well as interventions
        %exploring the possibility of treating incoming LTBI cases.
      \end{block}
    \end{column}
    
    \begin{column}{.3\textwidth}
      \begin{block}{TB Reduction Strategies for the United States}
        From the Hill model, we 
      \end{block}
      \begin{block}{Intervention Analysis}
        Intervention Analysis
        \begin{figure}[h]
          \begin{center}
            \includegraphics[scale=1]{incidencePlotRedEnLTBI.pdf}
          \end{center}
          \caption{A graph of the projected incidence levels per million in
                       the US-born, Foreign-born, and Total US population, given
                       that LTBI rates in Foreign-born arrivals are reduced by 
                       0\%, 50\%, 75\%, and 100\%.}
          \label{fig:redEnLTBI_incidence}
        \end{figure}
      \end{block}
      \begin{block}{Economic Modeling}
        We extended our basic implementation of the Hill model to incorporate
        economic data of treatment costs for Active and Latent Tuberculosis.  
%        In our model, we estimate the average heath care costs to be \$14,014.90
%        and \$403.45 to treat a single case of Active or Latent TB respectively.  
%        Given these costs and the average treatment rates for Active and Latent
%        TB in the USA, we modeled the expected economic burden of Tuberculosis
%        projected over the next hundred years.  

        \begin{figure}[h]
          \begin{center}
            \includegraphics[scale=1]{EnLTBIRedGroupCost.pdf}
          \end{center}
          \caption{A graph of the cumulative implementation costs, savings, and 
                       net US health care system costs of reducing rates of LTBI in
                       Foreign-born arrivals by 50\%, 75\%, and 100\%.  To determine the 
                       implementation cost for each intervention, the average cost
                       to identify and treat entering Foreign-born individuals with LTBI
                       was set to be \$600, \$800, and \$1000 for the 50\%, 75\%, 
                       and 100\% reduction strategies respectively.}
          \label{fig:redEnLTBI_costs}
        \end{figure}
      \end{block}
    \end{column}
    \begin{column}{.3\textwidth}
      \begin{block}{An Agent Based Implementation}
        We also wrote a stochastic agent-based version of the Hill model in both
        NetLogo and C++.
        \begin{block}{}
          \begin{column}{.45\textwidth}
            \begin{figure}[h]
              \begin{center}
                \includegraphics[width=\textwidth]{NLHMinc}
              \end{center}
              \caption{Incidence/million for R and NetLogo models (12 runs, $\Delta t$ = 0.1, popConst = 100)}
              \label{fig:NLHMinc}
            \end{figure}
          \end{column}
          \begin{column}{.45\textwidth}
            \begin{figure}[h]
              \begin{center}
                \includegraphics[width=\textwidth]{finalRunSmall}
              \end{center}
              \caption{Incidence/million for R and C++ models (2100 runs, $\Delta t$ = 0.01, popConst = 1)}
              \label{fig:finalRun}
            \end{figure}
          \end{column}
        \end{block}
      \end{block}
      
      \begin{block}{Stochastic Models as a Measure of Variability}
        The stochastic model gives us a sense of the variability of the results
        of the deterministic model.
        \begin{block}{}
          \begin{column}{.45\textwidth}
            \begin{figure}[h]
              \begin{center}
                \includegraphics[width=\textwidth]{IN0dist}
              \end{center}
              \caption{Distribution of USB Incidence (C++) with fitted Normal curve}
              \label{fig:IN0dist}
            \end{figure}
          \end{column}
          \begin{column}{.45\textwidth}
            \begin{figure}[h]
              \begin{center}
                \includegraphics[width=\textwidth]{IN1dist}
              \end{center}
              \caption{Distribution of FB Incidence (C++) with fitted Normal curve}
              \label{fig:IN1dist}
            \end{figure}
          \end{column}
        \end{block}
      \end{block}
      
      \begin{block}{Future Extensions}
        Possible extensions to our model include the incorporation of non-homogeneous
        contact structures and MDR (multi-drug resistant) TB.
      \end{block}
      
    \end{column}
  \end{columns}
\end{frame}
\end{document}
