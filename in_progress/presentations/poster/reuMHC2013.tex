\documentclass[final]{beamer}
%Poster Packages: 
\usepackage[orientation=lanscape,size=a0]{beamerposter}
%General Packages:
\usepackage{amsmath,amsthm, amssymb, latexsym}
%TikZ:
\usepackage{tikz}
\usetikzlibrary{shapes,arrows}
\input{arrowsnew}

%Theme:
\mode<presentation>{\usetheme{I6pd2}}

\graphicspath{{figures/}}

\newlength{\columnheight}
\setlength{\columnheight}{105cm}

\title{\huge Modeling Intervention Strategies for TB Control in the United States}
\author{Jessica Ginepro, Emma Hartman, Ryo Kimura, Matthew McDermott, Colin
        Pawlowski, \& Dylan Shepardson}
\institute[MHC]{Mathematical Modeling Group, Mount Holyoke College, South
                Hadley, MA, USA}
\date[July 31, 2013]{July 31, 2013}
\begin{document}
%For the compartmental models:
\tikzstyle{compartment} = [rectangle, draw, fill=blue!20, text badly centered,
  node distance=3.8em, inner sep=0.4em, rounded corners]
\tikzstyle{compartment rate} = [right of=start, color=black, node distance=2.2em,
text width=3em]
%TODO: Increase arrow head size
\tikzstyle{line} = [draw, -latexnew, arrowhead = 7mm]


\begin{frame}
  \begin{columns}
    \begin{column}{.3\textwidth}
      \begin{block}{Introduction}
        The most common epidemiological models are compartmental models, in
        which a system of differential equations describes how parts of the
        population move between different disease states, or
        \emph{compartments}. The canonical example of such a model is the SIR
        model, which has three compartments: Susceptible, Infected, and
        Recovered. The dynamics of a population modeled by this framework are
        shown in figure~\ref{fig:SIRFlowchart}, with associated system of
        differential equations~\ref{eq:SIRdes}.
        %TODO equalize figure heights
        \vspace{-3em}
        \begin{block}{}
          \begin{column}{.39\textwidth}
            \begin{figure}[h]
              \begin{center}
                \begin{tikzpicture}
                  \node [compartment] (S) {Susceptible (S)};
                  \node [compartment, below of=S] (I) {Infected (I)};
                  \node [compartment, below of=I] (R) {Recovered (R)};
                  \path [line] (S) -- node [compartment rate] {$\beta S(t)I(t)$} (I);
                  \path [line] (I) -- node [compartment rate] {$\gamma I(t)$} (R);
                \end{tikzpicture}
              \end{center}
              \caption{This flowchart depicts the standard SIR epidemiological
                       model. It is accomponanied by the system of differential
                       equations~\ref{eq:SIRdes}.}
              \label{fig:SIRFlowchart}
            \end{figure}
          \end{column}
          \begin{column}{.39\textwidth}
            \begin{figure}[h]
              \begin{align*}
                \frac{dS}{dt} &= -\beta S(t)I(t) \\
                \frac{dI}{dt} &= \beta S(t)I(t) - \gamma I(t) \\
                \frac{dR}{dt} &= \gamma I(t)\\
                N             &= S(t) + I(t) + R(t)\\
              \end{align*}
              \caption{The system of differential equations governing the SIR
                       model.}
              \label{eq:SIRdes}
            \end{figure}
          \end{column}
        \end{block}
        The SIR system, though very informative about basic disease dynamics,
        fails to capture the intricacies of more complicated diseases. This
        project examines the spread of tuberculosis (TB) in the United States
        (US) via compartmental models and stochastic, agent-based models. These
        explorations allow one to explore the impact, both epidemiological and
        economic, of various intervention strategies. These models are based on
        a 2012 compartmental model of TB in the US created by A. N. Hill, J. E.
        Becerra, and K. G. Castro.
      \end{block}
      \begin{block}{The Basic Hill Model}
        The Hill model is another compartmental model, illustrated in
        figure~\ref{fig:hillFlow}.
        \vspace{-2em}
        \begin{block}{}
          \begin{column}{.2\textwidth}
            \begin{figure}[h]
              \begin{center}
                \includegraphics[scale=.25]{HillModelFlowChart}
              \end{center}
              \caption{A flow chart representing the compartments of the Hill
                       Model.}
              \label{fig:hillFlow}
            \end{figure}
          \end{column}
          \begin{column}{.6\textwidth}
            The illustration of this model is not full, as there are two
            subpopulations in the Hill Model. TB dynamics in the US differ
            radically between US born (USB) individuals and foreign born (FB)
            individuals,  so this model maintains two distinct populations in
            which the disease can spread. There is a near identical diagram to
            describe the dynamics of the model for the FB population. Further,
            each compartment in this model also loses people to do natural
            death. The system of differential equations for this model is
            naturally more complex and is provided in an appendix. In order to
            implement this model, Hill, Becerra, and Castro used the \texttt{R}
            programming language, with the \texttt{lsoda} routine to numerically
            solve the differential equations. 
          \end{column}
        \end{block}
        For this implementation, various numerical solvers were implemented,
        including an Eulerian Method, a Quadratic Method, a Fourth-order runge
        kutta method, and the $\texttt{lsoda}$ routine. Once the basic Hill
        model was implemented, further tracking capabilities were added as well
        as economic modeling. This implementation tracks total active TB cases,
        entering cases of LTBI, and treatment costs for active and latent disease
        for all populations. Tracking was used to evaluate various interventions
        explored in Hill, Becerra, and Castro's work, as well as interventions
        exploring the possibility of treating incoming LTBI cases.
      \end{block}
    \end{column}
    \begin{column}{.3\textwidth}
      \begin{block}{TB Reduction Strategies for the United States}
        From the Hill model, we 
      \end{block}
      \begin{block}{Intervention Analysis}
        Intervention Analysis
        \begin{figure}[h]
          \begin{center}
            \includegraphics[scale=1]{incidencePlotRedEnLTBI.pdf}
          \end{center}
          \caption{caption here!}
          \label{fig:redEnLTBI_costs}
        \end{figure}
      \end{block}
      \begin{block}{Economic Modeling}
        We extended our basic implementation of the Hill model to incorporate
        economic data of treatment costs for Active and Latent Tuberculosis.  
        In our model, we estimate the average heath care costs to be \$14,014.90
        and \$403.45 to treat a single case of Active or Latent TB respectively.  
        Given these costs and the average treatment rates for Active and Latent
        TB in the USA, we modeled the expected economic burden of Tuberculosis
        projected over the next hundred years.  

        \begin{figure}[h]
          \begin{center}
            \includegraphics[scale=1]{EnLTBIRedGroupCost.pdf}
          \end{center}
          \caption{A graph of the cumulative implementation costs, savings, and 
                       net US health care system costs of reducing rates of LTBI in
                       Foreign-born arrivals by 50\%, 75\%, and 100\%.  To determine the 
                       implementation cost for each intervention, the average cost
                       to identify and treat entering Foreign-born individuals with LTBI
                       was set to be \$600, \$800, and \$1000 for the 50\%, 75\%, 
                       and 100\% reduction strategies respectively.}
          \label{fig:redEnLTBI_costs}
        \end{figure}
      \end{block}
    \end{column}
    \begin{column}{.3\textwidth}
      \begin{block}{An Agent Based Implementation}
        Implementing an Agent Based Framework
      \end{block}
    \end{column}
  \end{columns}
\end{frame}
\end{document}
