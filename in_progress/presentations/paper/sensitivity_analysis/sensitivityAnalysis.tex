\documentclass{article}[11pt]
\usepackage{amsmath,amssymb,amsthm}
\usepackage{graphicx}
\graphicspath{{../figures/}}
\begin{document}
\title{Sensitivity Analysis}
\author{Colin Pawlowski}
\date{}
\maketitle

%This document would most likely end up in the Supplementary Material for the paper

\section{Overview}
We divide our sensitivity analysis into two sections, to analyze the robustness of the deterministic and stochastic models separately.  For the deterministic model, we used Latin Hypercube Sampling to test the sensitivity of our model to input parameters from the Hill model and estimates for treatment costs.  
For the stochastic model, we fixed all input parameters and estimates for treatment costs to default values, and ran the model for 2,000 trials.  

\section{Deterministic Model}
\subsection{Latin Hypercube Sampling}
Following the example of the Hill model, we generated a Latin Hypercube Sample varying 18 of the input parameters.  From these parameters, 16 are identical to the input parameters varied in the sensitivity analysis of the original Hill model, and the remaining two parameters, $C_{t}$ and $C_{l}$, are variables for the average cost of Active and Latent TB treatment in the US.  Probability distributions for the original 16 parameters of the Hill model were all set to be Triangular, with mode at the best fit value and end points at the 2.5 and 97.5 percentile values reported in the Hill model.  Probability distributions for $C_{t}$ and $C_{l}$ were set to be Uniform, with range +/- 10\% of the estimated value.  

With a random Latin Hypercube Sample of size n=100,000, we computed partial rank correlation coefficients (PRCC) for each of the initial parameters and treatment costs, according to four different outcomes: 1) projected annual incidence in 2100 in the overall population, 2) projected cumulative cost of Latent TB treatments by 2100, 3) projected cumulative cost of Active TB treatments by 2100, 4) projected cumulative total cost of TB treatments by 2100.  For outcome 1, PRCC values are shown alongside PRCC values computed in the original Hill model for the same outcome in Table 1, showing the closeness of our findings to the sensitivity results of the original Hill model.  PRCC values for the remaining outcomes are reported in Table 2.  


\begin{table}
\centering
\begin{tabular}{l r r}
\hline\hline\\
Parameter & Extended Hill Model & Original Hill model \\ [0.5ex]
\hline\\
$\sigma_{L}$  & -0.9303 & -0.9381 \\
$v^{L}_{1}$   & 0.7871  & 0.8309 \\
$f$                 & 0.7050  & 0.8072 \\
$p$                & 0.8369  & 0.6100 \\
$ARI_{0}$      & 0.5950  & 0.4939 \\
$q$                & 0.5797  & 0.4543 \\
$g$                & 0.6122  & 0.4517 \\
$\sigma_{F}$ & -0.4911 & -0.3772 \\
$r_{1}$          & 0.0028  & -0.1109 \\
$r_{0}$          & 0.0018  & 0.0760 \\
$\mu_{d}$      & 0.0923 & 0.0513 \\
$x$                 & 0.0999 & 0.0345 \\
$v^{L}_{0}$    & 0.0133 & 0.0266 \\
$\phi$             & 0.0082 & 0.0177 \\
$e_{0}$          & 0.0178 & -0.0072 \\
$e_{1}$          & 0.1154 & 0.0046 \\
$C_{t}$          & -0.0023 & N/A \\
$C_{l}$           & 0.0009 & N/A \\ [1ex]
\hline
\end{tabular}\\[1ex]

{\bf Table 1.} PRCC values for projected annual incidence in 2100
 in the overall population, alongside corresponding values from the
original Hill model.
\end{table}



\begin{table}
\centering
\begin{tabular}{l r r r}
\hline\hline\\
Parameter & Latent & Active & Total\\ [0.5ex]
\hline\\
$\sigma_{L}$  & 0.9612  & -0.9284 & 0.4169 \\
$v^{L}_{1}$   & -0.4190 & 0.6470  & 0.3533 \\
$f$                 & 0.7467  & 0.5493  & 0.8083 \\
$p$                & 0.3371  & 0.8810  & 0.8776 \\
$ARI_{0}$      & 0.3920  & 0.6728  & 0.7337 \\
$q$                & 0.3837  & 0.6573  & 0.7200 \\
$g$                & 0.1120  & 0.5369  & 0.5182 \\
$\sigma_{F}$ & -0.1138 & -0.5631 &-0.5435 \\
$r_{1}$          & 0.1253 &  0.0658  & 0.1401 \\
$r_{0}$          & 0.1325  & 0.0878  & 0.1658 \\
$\mu_{d}$      & 0.0249 & 0.0560  & 0.0613 \\
$x$                 & 0.0214 & 0.1282  & 0.1253 \\
$v^{L}_{0}$   & -0.3103 & 0.0502  &-0.1867 \\
$\phi$            & -0.0023 & -0.0102 &-0.0107 \\
$e_{0}$          & 0.0081 & 0.0253   & 0.0269 \\
$e_{1}$          & 0.0804 & 0.1653   & 0.1926 \\
$C_{t}$          & 0.0011 & 0.7024   & 0.6385 \\
$C_{l}$           & 0.8515 & 0.0013  & 0.7598 \\ [1ex]
\hline
\end{tabular}\\[1ex]

{\bf Table 2.} PRCC values for cumulative US Health Care system costs from 
Latent TB treatment, Active TB treatment, and Total treatment costs
\end{table}

%Add table of distributions for initial parameters and treatment costs
%Add table of Partial Rank Correlation Coefficients

\subsection{Variability due to Uncertainty of Treatment Costs}
Next, we performed more extensive multivariate analysis on the economic part of the deterministic model.  Keeping all of the input parameters from the Hill model at their default values, we ran the model approximately two million times, varying the cost of treatment for LTBI and Active TB and the treatment rate for incoming LTBI.  \\

%Add 3-D contour plot
%Add explanation of the effects of treatment costs, incoming LTBI treatment rate on total HCS cost

\subsection{Variability of Intervention Cost for Treating Cases of Incoming LTBI}
Similarly, we performed extensive multivariate analysis on the intervention cost for reducing incoming LTBI.  With all input parameters from the Hill model at their default values, we ran the model approximately two million times, varying the treatment rate, incidence, and cost to cure for incoming LTBI.  \\

%Add 3-D contour plot
%Add explanation of the effects of intervention costs, incoming LTBI treatment rate on total HCS cost

\section{Stochastic Model}
\subsection{Variability due to Probabilistic Nature of Disease Spread}

\section{Summary}
\end{document}
